% SCOPE CONTRACT: EMPIRICAL ANALYSIS (Section 6)
% Content derived from Empirical Resilience Research (Helium, Geodnet, Hivemapper)

\section{Empirical Stress Analysis}
\label{sec:empirical_analysis}

\subsection{Introduction}
Decentralized Physical Infrastructure Networks (DePIN) represent a fundamental shift in the deployment and maintenance of critical real-world infrastructure, leveraging cryptographic incentives to coordinate distributed hardware provision across sectors ranging from wireless connectivity to geospatial mapping \cite{FrontiersDePIN2025}. Unlike purely digital assets within the Decentralized Finance (DeFi) ecosystem, DePIN protocols introduce a complex layer of physical constraints—including hardware sunk costs, geographic friction, and logistical latency—that fundamentally alters their economic risk profile \cite{RapidInnovation2024}. Consequently, the sustainability of these networks is not merely a function of code, but of the intricate interplay between tokenomic mechanism design and the behavioral economics of hardware operators under stress \cite{HackMD2025}.

Historically, the evaluation of such systems has relied heavily on Agent-Based Modeling (ABM) and stochastic simulations to forecast network behavior \cite{RapidInnovation2024}. While valuable for exploring theoretical bounds, simulations often operate on rational agent assumptions that fail to capture the nuances of human panic, the "stickiness" of physical deployments, or the unpredictability of exogenous market shocks. There is a growing recognition within the research community that empirical analysis—grounded in the historical performance of live networks under actual stress—offers a more rigorous and actionable methodology for assessing resilience \cite{Coincub2025}.

This section establishes a comprehensive methodology for comparative DePIN analysis without the use of simulation. By operationalizing the "Event Study Methodology" utilized in financial econometrics \cite{Hacken2024}, we construct a static stress-testing framework that benchmarks the Onocoy network against the historical performance of mature Solana-based DePIN peers, including Helium, Hivemapper, Render, and Geodnet. This approach replaces probabilistic forecasting with deterministic observation, analyzing how different incentive structures—such as Burn-and-Mint Equilibrium (BME) versus Capped Supply—have historically responded to demand contractions, liquidity crises, and competitive yield pressures \cite{FrontiersDePIN2025}.

The objective is to provide a diagnostic tool that is equally valuable for the academic thesis and the broader DePIN community. By isolating specific "Failure Modes" such as Reward-Demand Decoupling and Latent Capacity Degradation \cite{RapidInnovation2024}, and mapping them to observable on-chain metrics, this report offers a nuanced framework for evaluating whether a protocol’s incentive mechanism can retain providers, sustain service capacity, and prevent economic collapse in the absence of speculative growth.

\subsection{Theoretical Framework: Operationalizing Stress Without Simulation}
To effectively analyze DePIN robustness without the generative capabilities of simulation, one must adopt a framework based on Comparative Statics and Empirical Event Studies. This involves defining "stress" not as a variable in a model, but as a specific set of historical market conditions (regimes) and analyzing the elasticity of network supply and demand in response to those conditions \cite{PMC2022}.

\subsubsection{The "Disease": Defining Stress Factors in Physical Networks}
In the context of this static analysis, stress factors are exogenous shocks that challenge the economic viability of the network. We categorize these stressors into three primary buckets that determine long-term sustainability \cite{RapidInnovation2024}:

\begin{itemize}
    \item \textbf{The Subsidy Gap (Reward Addiction):} This represents the structural deficit between the real operating costs (OpEx) plus capital expenditures (CapEx) of infrastructure providers and the actual fiat revenue entering the system. In early-stage networks, this gap is bridged by speculative token incentives. A critical stress point occurs when token prices decline, widening this gap and forcing providers to operate at a loss. Static analysis measures this by tracking the historical divergence between "Real Yield" (burn-based revenue) and "Dilutive Yield" (inflationary rewards) over time \cite{RapidInnovation2024}.
    \item \textbf{Speculative Fragility:} This metric defines the correlation between network security (provider uptime/retention) and token price volatility. High fragility implies that a drop in token price triggers immediate and proportional infrastructure churn. In a non-simulated framework, this is measured by calculating the "Beta" of active node counts relative to the token's price drawdown during specific bear market windows (e.g., the 2022-2023 crypto winter) \cite{RapidInnovation2024}.
    \item \textbf{Competitive Yield Pressure:} This refers to the elasticity of provider participation when alternative networks offer higher returns for similar hardware. As DePIN hardware becomes more generalized (e.g., GPUs for Render vs. io.net, or GNSS receivers for Geodnet vs. Onocoy), the switching costs decrease, increasing the risk of "vampire attacks" where supply migrates instantly to the highest bidder \cite{RapidInnovation2024}.
\end{itemize}

\subsubsection{The "Cure": Mechanism Design as a Static Defense}
If stress is the disease, the tokenomic mechanism is the immune system. In a static analysis, we evaluate the architectural capacity of different mechanisms to absorb shock. The primary designs observed in the Solana DePIN ecosystem include:

\begin{itemize}
    \item \textbf{Burn-and-Mint Equilibrium (BME):} Utilized by Helium and Render, this dual-token model requires users to burn the native token to create non-transferable data credits (fixed in fiat value). This theoretically caps supply based on demand \cite{RenderBME}. The static stress test involves calculating the "Equilibrium Price"—the token price at which the burn rate from usage exactly offsets the mint rate for provider rewards—and comparing it to the actual market price.
    \item \textbf{Capped Supply with Emissions Decay:} Employed by Onocoy (and Bitcoin), this model enforces scarcity via a hard cap and diminishing returns (halvings) \cite{OnocoyToken}. The stress test here focuses on the "Security Budget Analysis"—determining whether the diminishing block reward remains sufficient to cover aggregate provider OpEx in the absence of significant fee revenue.
    \item \textbf{Sunk Cost Moats:} A crucial differentiator for DePIN is that the hardware itself often acts as a "staked" asset. The analysis must quantify "Economic Friction"—the financial loss incurred by a provider who exits the network due to the illiquidity of the hardware. This sunk cost acts as a retention buffer, often more effective than on-chain staking during liquidity crises \cite{RapidInnovation2024}.
\end{itemize}

\subsubsection{Methodology: Empirical Event Studies}
To replace simulation, we utilize the Event Study Methodology, a standard approach in financial econometrics used to measure the impact of specific events on asset value and participant behavior \cite{Hacken2024}. By treating historical market shocks—such as the FTX collapse, Helium's migration to Solana, or Hivemapper's reward restructuring—as distinct "events," we can quantify the abnormal churn or abnormal returns associated with these stressors.

This method allows us to infer the future behavior of a network like Onocoy by benchmarking it against the past empirical behavior of comparable networks under similar conditions. We look for "Abnormal Churn Rates"—deviations from the expected baseline of provider exits—following negative price shocks to determine the true "Elasticity of Provider Exit" \cite{RapidInnovation2024}.

\subsection{Comparative Scope: The Solana DePIN Ecosystem}
Solana has emerged as the dominant execution layer for DePIN due to its high throughput, low transaction costs (\$0.00025), and state compression capabilities, which allow for the cost-effective management of millions of physical nodes \cite{Coincub2025}. To answer the thesis questions regarding robustness, we establish a comparative set of key projects representing different hardware profiles and stress responses.

\subsubsection{The Comparative Set}
We categorize the projects into three distinct "archetypes" based on their hardware cost profile and service model, as these factors dictate their response to economic stress:

\paragraph{Archetype A: Commodity Sensor Networks (Low OpEx, Moderate CapEx)}
\begin{itemize}
    \item \textbf{Helium (IoT/Mobile):} The mature market leader with over 1 million hotspots. It serves as the primary case study for long-term retention behavior and the transition from L1 to Solana \cite{Coincub2025}.
    \item \textbf{Geodnet:} A direct competitor to Onocoy in the GNSS/RTK sector. Focused heavily on stable B2B revenue and industrial utility, utilizing "Location NFTs" to manage density \cite{GeodnetIP7}.
    \item \textbf{Onocoy:} The thesis subject. An emerging GNSS network with a distinct governance approach and capped token supply model \cite{RapidInnovation2024}.
\end{itemize}

\paragraph{Archetype B: High-Performance Compute (High OpEx, High CapEx)}
\begin{itemize}
    \item \textbf{Render:} Decentralized GPU rendering focused on the creator economy. High electricity costs make these providers highly sensitive to token price drops, serving as a proxy for "mercenary" capital behavior \cite{NewtonRender}.
    \item \textbf{io.net:} Aggregated GPU compute for AI/ML workloads. A newer entrant experiencing massive growth but untested in a prolonged bear market, dealing with potential supply shocks from "cluster" migration \cite{IoNet2025}.
\end{itemize}

\paragraph{Archetype C: "Proof of Physical Work" (Active Labor)}
\begin{itemize}
    \item \textbf{Hivemapper:} Dashcam mapping. Unlike passive sensor networks, this requires active human labor (driving). Incentives must cover fuel and time, creating a higher churn risk during price downturns \cite{CoinMarketCapHM}.
\end{itemize}

\subsubsection{Metric Standardization for Static Comparison}
To compare these diverse networks without simulation, we standardize specific metrics that serve as proxies for health and stress resilience \cite{Coincub2025}.

\begin{table}[ht]
    \centering
    \small
    \renewcommand{\arraystretch}{1.5}
    \begin{tabular}{|p{0.23\linewidth}|p{0.27\linewidth}|p{0.42\linewidth}|}
        \hline
        \textbf{Category} & \textbf{Standardized Metric} & \textbf{Definition \& Proxy Utility} \\
        \hline
        Incentive Solvency & Burn-to-Mint Ratio & The ratio of tokens burned (revenue) to tokens minted (incentives). A value $< 1$ indicates subsidy dependence; $> 1$ indicates sustainable deflation \cite{Coincub2025}. \\
        \hline
        Provider Retention & 30-Day Node Retention & Percentage of nodes active at $T+30$ days compared to $T0$, specifically analyzed following a token price drop of $>20\%$ \cite{RapidInnovation2024}. \\
        \hline
        Economic Efficiency & Revenue per Node & Total Network Revenue divided by Total Active Nodes. Indicates whether the average provider is economically viable without speculative token rewards \cite{AlloraResearch}. \\
        \hline
        Speculative Velocity & Token Turnover & Daily Trading Volume divided by Circulating Market Cap. High turnover during price drops often signals "mercenary capital" exit rather than utility usage \cite{GateSquare2025}. \\
        \hline
        Valuation Risk & FDV / Annualized Revenue & Fully Diluted Valuation divided by Annualized Revenue. A comparative valuation metric to assess the speculative premium embedded in the token price \cite{HeliumMobileCMC}. \\
        \hline
    \end{tabular}
    \caption{Standardized Metrics for Comparison}
    \label{tab:comparative_metrics}
\end{table}

\subsection{Empirical Stress Test: Historical Performance Analysis}
This section executes the "static stress test" by analyzing how the comparative set performed during documented historical stress events. This empirical data replaces the need for generating synthetic stress scenarios in a simulation.

\subsubsection{Stress Scenario 1: The Liquidity Shock \& Crypto Winter (2022-2023)}
\textbf{Context:} During the 2022 crypto winter, the market contracted significantly. Helium (HNT) saw its price decline from highs of $\sim\$55$ to under $\$2$, losing approximately $95\%$ of its value \cite{BitDegreeHelium}. This serves as a perfect empirical case study for Liquidity-Driven Incentive Compression \cite{RapidInnovation2024}.

\textbf{Helium's Response:}
\begin{itemize}
    \item \textbf{Price Impact:} Despite the catastrophic price drop, the number of active hotspots did not collapse proportionally. The physical nature of the miners (mounted on roofs with sunk costs of $\sim\$500$) created high exit friction \cite{RapidInnovation2024}.
    \item \textbf{Provider Behavior:} Providers largely remained online because the marginal cost of operation (electricity) was negligible compared to the effort of uninstallation. This validates the "Sunk Cost Moat" theory: hardware creates resilience against immediate churn but stalls future capacity growth under stress.
    \item \textbf{Failure Mode Observed:} While retention remained high, \textit{Latent Capacity Degradation} occurred. New deployments flatlined, and maintenance of existing nodes likely dropped (though harder to measure without simulation, forum sentiment confirms this), degrading actual coverage quality over time \cite{GateSquare2025}.
    \item \textbf{Governance Reaction:} Helium proposed the migration to Solana and the implementation of sub-DAOs (IOT and MOBILE) to compartmentalize risk—a clear example of the "Emergency Centralization" or "Narrative Pivot" archetype \cite{RapidInnovation2024}.
\end{itemize}

\textbf{Implication for Onocoy:} Onocoy can expect its existing GNSS stations to be "sticky" during price crashes due to high installation effort (antennas require precise placement). However, new station growth will freeze. Resilience strategies should focus on maximizing the utility of existing capacity rather than incentivizing new hardware during these periods.

\subsubsection{Stress Scenario 2: The "Vampire Attack" \& Competitive Yield (2024-2025)}
\textbf{Context:} Emerging DePIN protocols often target the same provider base. In the GNSS sector, Geodnet and Onocoy compete for the same rooftop real estate and technically savvy installers \cite{TripleMining2025}.

\textbf{Geodnet vs. Onocoy Dynamics:}
\begin{itemize}
    \item \textbf{Yield Sensitivity:} Geodnet focuses on stable B2B revenue (RTK services) and utilizes a "Location NFT" to limit supply in saturated areas, enforcing a meritocracy that protects individual miner yield \cite{GeodnetIP7}.
    \item \textbf{Elastic Provider Exit:} Multi-mining (running both Geodnet and Onocoy on similar hardware) is possible but introduces complexity. Providers demonstrate "mercenary loyalty," prioritizing the network offering better immediate liquidity or yield.
    \item \textbf{Observed Behavior:} When Geodnet token (GEOD) prices rallied or burns increased due to enterprise contracts, it attracted professional surveyors who prioritize stable fiat-equivalent income over speculative tokens \cite{GeodnetLocationNFT}.
    \item \textbf{Failure Mode:} \textit{Elastic Provider Exit Under External Yield Pressure} \cite{RapidInnovation2024}. If Onocoy's rewards lag significantly behind Geodnet's for prolonged periods, "dual-mining" setups may convert to "single-mining" for the competitor to optimize bandwidth or hardware stability.
\end{itemize}

\subsubsection{Stress Scenario 3: Operational Cost Shock (Hivemapper 2024)}
\textbf{Context:} Unlike stationary miners, Hivemapper requires active driving, introducing variable costs (fuel, time). This creates a higher OpEx floor, making the network more sensitive to token price drops.

\textbf{Hivemapper's Stress:}
\begin{itemize}
    \item \textbf{Cost Sensitivity:} When the HONEY token price dipped, the "real wage" for driving dropped below the cost of fuel/time for many casual mappers.
    \item \textbf{Churn Response:} Unlike Helium hotspots, Hivemapper contributors stop mapping immediately when incentives fall below the effort threshold. Active contributors declined or plateaued during price stagnation periods, illustrating \textit{Profitability-Induced Provider Churn} \cite{Coincub2025}.
    \item \textbf{Mitigation Strategy:} Hivemapper responded by pivoting to B2B enterprise deals (Map Credits) to burn tokens and support price, and by introducing "Honey Bursts" (targeted bonuses) to surge supply only where needed—a form of "Incentive Re-Targeting" \cite{RapidInnovation2024}.
\end{itemize}

\subsection{Tokenomic Mechanism Analysis: Comparative Robustness}
We now analyze the specific tokenomic parameters of Onocoy against the industry standards established by the comparative set. This "static" analysis evaluates the theoretical soundness of the design mechanics.

\subsubsection{The Burn-and-Mint Equilibrium (BME)}
Most major Solana DePINs (Helium, Render, Hivemapper) utilize BME to stabilize token value against usage.
\begin{itemize}
    \item \textbf{Mechanism:} Users pay in Fiat/Stablecoins $\to$ System buys \& burns Tokens $\to$ Providers mint new Tokens.
    \item \textbf{Stress Behavior:} In a bull market, BME is reflexive (usage burns supply $\to$ price goes up $\to$ usage costs less tokens). In a bear market, if usage is low, the "Mint" (inflation) outweighs the "Burn," creating a death spiral of dilution \cite{MediumTokenomics2026}.
\end{itemize}

\textbf{Onocoy's Variance:} Onocoy utilizes a "capped supply" model with deflationary elements (buyback and burn from revenue) rather than a pure BME where minting is unlimited to meet demand \cite{OnocoyToken}.

\textbf{Assessment:} Onocoy's capped supply is theoretically more resilient to hyperinflationary death spirals than pure BME. However, it is less flexible in subsidizing early growth if the token price is too high (scarcity can stifle adoption) or too low (insufficient reward budget to attract miners).

\subsubsection{Incentive Solvency: The Burn-to-Mint Ratio}
This is the "Golden Metric" for static analysis, serving as the primary indicator of long-term solvency: $\text{Ratio} = \frac{\text{Burn}}{\text{Mint}}$.

\begin{itemize}
    \item \textbf{Helium Mobile:} Has historically approached a 1:1 or deflationary ratio in specific months due to high subscriber revenue, indicating high solvency \cite{HeliumMobileCMC}.
    \item \textbf{Hivemapper:} Historically low ratio (high emissions, low burn), though improving with enterprise deals \cite{CoinMarketCapHM}.
    \item \textbf{Onocoy:} As an early-stage network, this ratio is likely $< 0.1$ (highly subsidized).
\end{itemize}

\textbf{Risk Assessment:} A ratio consistently below 1.0 implies the network is consuming its own market capitalization to survive. Long-term solvency requires this metric to cross 1.0. The comparative data suggests that networks which fail to improve this ratio within 24 months of launch face significant risk of "Reward Addiction" failure.

\subsection{DePIN Failure Modes: A Diagnostic Matrix}
Based on the historical data and theoretical framework, we define five operational failure modes for Onocoy and other DePINs. These definitions allow for "diagnosis" without the need for complex simulation \cite{RapidInnovation2024}.

\begin{enumerate}
    \item \textbf{Reward-Demand Decoupling:} Emissions persist at a high rate while network usage (demand) flatlines or drops. If GNSS stations deploy faster than RTK data buyers can be onboarded, ONO rewards will dilute rapidly \cite{Coincub2025}.
    \item \textbf{Liquidity-Driven Incentive Compression:} External market sell-offs reduce the fiat value of the fixed token rewards, pushing provider ROI below the break-even point. GNSS stations are low-power (low OpEx), so they are less sensitive to this than GPU networks but more sensitive than passive staking protocols \cite{RenderThesisLB}.
    \item \textbf{Latent Capacity Degradation:} Providers stop maintaining hardware due to low rewards. High "Active Node" count but falling "Quality/Uptime" scores. RTK requires precise calibration; if rewards drop, providers may neglect maintenance, degrading network accuracy \cite{GateSquare2025}.
    \item \textbf{Elastic Provider Exit:} Capital (hardware) moves instantly to a competitor offering higher yield. Significant risk if Geodnet offers higher yield for compatible hardware.
    \item \textbf{The Death Spiral:} Feedback loop where Price Drop $\to$ Lower Yield $\to$ Miner Churn $\to$ Lower Security $\to$ Lower Demand $\to$ Further Price Drop \cite{MediumHeliumMiner}.
\end{enumerate}

\subsection{Conclusion: Benchmarking Summary}
Analyzing DePIN tokenomics without simulation is not only possible but often more revealing than theoretical modeling. By employing a comparative event study framework, we can observe how real-world networks like Helium, Hivemapper, and Render have responded to the stress of bear markets, liquidity crunches, and competitive pressure.

The empirical data indicates that physical sunk costs act as a powerful retention buffer, preventing immediate collapse during price drops (unlike DeFi). However, this resilience is temporary; without a transition to Real Yield (Burn-to-Mint ratios $> 1$), networks eventually succumb to Reward Addiction and Latent Capacity Degradation.

For Onocoy, this comparison highlights the need to tighten the coupling between Rewards and Realized Demand, and to differentiate incentives to reward high-commitment, professional providers—increasing the economic moat against "mercenary" switching.
