\documentclass[12pt,a4paper]{article}

% Import preamble
% Character encoding and languages
\usepackage[utf8]{inputenc}
\usepackage[T1]{fontenc}
\usepackage[english]{babel}

% Fonts: Time-tested academic choices (Times-like)
% \usepackage{mathptmx} % Use Times font for text and math
\usepackage{courier}  % Courier for monospace
\usepackage{helvet}   % Helvetica for sans-serif

% Page Layout
\usepackage[a4paper, top=3cm, bottom=3cm, left=3cm, right=3cm]{geometry}

% Line Spacing (1.5 spacing is standard for review/readability)
\usepackage{setspace}
\onehalfspacing

% Math Support
\usepackage{amsmath}
\usepackage{amssymb}
\usepackage{amsfonts}
\usepackage{mathtools}

% Graphics and Figures
\usepackage{graphicx}
\usepackage{float}
\usepackage[font=small,labelfont=bf]{caption}
\usepackage{subcaption}

% Tables
\usepackage{booktabs} % Professional table rules
\usepackage{array}
\usepackage{multirow}
\usepackage{longtable}
\usepackage{tabularx} % Auto-width tables

% Bibliography management
\usepackage[backend=biber, style=numeric, sorting=none]{biblatex}
\addbibresource{bibliography.bib}

% Utility
\usepackage{enumerate} % Easy list customization
\usepackage[pdftex, pdfusetitle]{hyperref}
\usepackage{url}


% Hyperref setup (clean, no boxes)
\hypersetup{
    colorlinks=true,
    linkcolor=black,
    citecolor=black,
    filecolor=black,
    urlcolor=black,
    pdfborder={0 0 0}
}

% Fix URL formatting in bibliography
\urlstyle{same} % Use the same font as text for URLs (avoids wide Courier)

% Allow URLs to break on lowercase and uppercase letters to prevent bad spacing
\setcounter{biburllcpenalty}{7000}
\setcounter{biburlucpenalty}{8000}


% Section Numbering Depth
\setcounter{secnumdepth}{3}
\setcounter{tocdepth}{3}

\title{
    \textbf{DePIN Tokenomics Under Stress} \\
    \large A Simulation-Based Sustainability Analysis Using the Onocoy Network as a Case Study
}
\author{Author A \\ Author B \\ Author C}
\date{\today}

\begin{document}

% Title Page
\begin{titlepage}
    \centering
    \vspace*{1cm}
    
    {\Huge \textbf{DePIN Tokenomics Under Stress}}
    
    \vspace{0.5cm}
    {\Large A Simulation-Based Sustainability Analysis Using the Onocoy Network as a Case Study}
    
    \vspace{1.5cm}
    
    \textbf{Author A} \\
    \textbf{Author B} \\
    \textbf{Author C}
    
    \vfill
    
    CAS Blockchain \\
    Lucerne University of Applied Sciences and Arts (HSLU)
    
    \vspace{0.8cm}
    
    \today
    
\end{titlepage}

% Front Matter
\pagenumbering{roman}

% Abstract
\section*{Abstract}
Decentralized Physical Infrastructure Networks (DePIN) rely on token-based incentive systems to bootstrap and sustain real-world infrastructure operated by independent providers. While tokenomics is frequently discussed in terms of growth and market valuation, significantly less attention has been paid to how these mechanisms behave under adverse conditions such as demand shocks, liquidity events, or deteriorating provider economics \cite{Messari2024}. This gap is particularly consequential for DePIN systems, where infrastructure churn can lead to irreversible losses in service capacity due to physical deployment constraints.

This thesis evaluates the robustness of DePIN tokenomic mechanisms under standardized stress conditions using a reproducible, simulation-based framework. Rather than forecasting token prices or predicting network success, the analysis focuses on comparative, directional behavior: how different incentive structures respond when subjected to adverse scenarios \cite{Morris2019}. An agent-based simulation model is employed to represent heterogeneous infrastructure providers interacting with a rule-based protocol under exogenous demand and macroeconomic regimes \cite{Braakman2022}.

The framework is applied to the Onocoy network and its ONO token as a primary case study. Results are interpreted as conditional insights into mechanism resilience, not as claims of causal certainty or real-world prediction. The contribution of this work lies in demonstrating how stress-testing can be systematically applied to DePIN tokenomics, offering a tool for researchers and builders to evaluate incentive robustness before failure occurs.
\vspace{1cm} 
% End of Abstract

\newpage

% Table of Contents
\tableofcontents
\newpage

% List of Figures
\listoffigures
\newpage

% List of Tables
\listoftables
\newpage

% Main Body
\pagenumbering{arabic}

% Include sections
% SCOPE CONTRACT: PERSON A (Foundations)
% MUST:
% - Define DePIN specific context and terminology
% - Define Stress Testing methodologies in financial/crypto systems
% - Establish the academic framework for the study
% - Provide literature review summary relevant to tokenomics
% MUST NOT:
% - Detail the specific tokenomics mechanics of the target protocol (Person B)
% - Present simulation results or specific modeling data (Person C)
% - Include heavy mathematical formulas for the specific model (Person C)

\section{Foundations}
\label{sec:foundations}

% Content for Person A goes here

\newpage

% SCOPE CONTRACT: PERSON B (Theoretical Framework)
% Content: General Tokenomic Theory, BME Definitions, Solana Ecosystem Context

\section{Theoretical Tokenomic Framework}
\label{sec:theoretical_framework}

\subsection{Introduction}
[TODO: Define the scope of Tokenomics in DePIN. Key themes: Sustainability, Mechanism Design, and the 'Physical' constraint.]

\subsection{DePIN as a Cyber-Physical System}
\label{subsec:depin_cps}
This research defines Decentralized Physical Infrastructure Networks (DePIN)—and specifically the Onocoy network—not merely as cryptocurrency projects, but as Cyber-Physical Systems (CPS). In this framework, physical hardware (GNSS reference stations) and digital state (token incentives) operate in dynamic, continuous feedback loops. Within this loop, the token functions as what Voshmgir describes as a "Purpose-Driven Token"—a steering signal designed to coordinate collective action (infrastructure deployment) that traditional market mechanisms fail to bootstrap efficiently \cite{Voshmgir2020}. This definition draws directly from the foundational work of Michael Zargham (BlockScience), who established "Token Engineering" as a discipline rooted in Cyber-Physical Systems Engineering rather than speculative finance.

BlockScience, founded by Zargham, has established the industry standard for modeling complex cryptoeconomic systems, having successfully applied these principles to the Ethereum gas market, The Graph, and most notably, Filecoin. % TODO-CITE: [Zargham/BlockScience] Find specific case study for Ethereum/Graph modeling
Their work on Filecoin is particularly relevant as a comparative baseline for Onocoy, given the shared challenge of incentivizing massive-scale physical infrastructure deployment.

Aligning with this industry standard, this thesis adopts a "Digital Twin" simulation methodology. Just as BlockScience utilizes cadCAD to create high-fidelity simulations of network behavior, this study employs a specialized "low-fidelity Digital Twin"—the DePIN Tokenomic Stress Evaluator (DTSE). This approach moves beyond simple price forecasting to model the structural feedback loops between hardware CAPEX, provider retention, and token variance. As demonstrated by Bernardineli's work on Filecoin's minting policies, such control-theoretic modeling is essential for identifying failure modes in systems where physical hardware constraints create latency in network response \cite{Bernardineli2022}.

\subsection{Comparative Scope: The Solana DePIN Ecosystem}
\label{subsec:solana_scope}
This section analyzes the tokenomic design patterns of eight leading DePIN protocols on Solana to establish a baseline for industry standards.

\subsubsection{Project Archetypes}
[TODO: Analyze the following projects based on Hardware Sunk Cost vs. Active Work:]
\begin{enumerate}
    \item \textbf{Helium (IoT/Mobile):} The Burn-and-Mint Standard.
    \item \textbf{Hivemapper (Mapping):} Active labor constraints.
    \item \textbf{Render (Compute):} High-end GPU economics.
    \item \textbf{io.net (Aggregated Compute):} Cluster economics.
    \item \textbf{Geodnet (GNSS):} Location NFT and hex-based capping.
    \item \textbf{Grass (Bandwidth):} Zero-marginal cost participation.
    \item \textbf{Nosana (CI/CD):} Compute grid incentives.
    \item \textbf{Aleph.im (Storage/Compute):} Infrastructure persistence.
\end{enumerate}

\subsection{Tokenomic Building Blocks}
\label{subsec:building_blocks}
\subsubsection{Emission Schedules \& Inflation}
[TODO: Compare 'Halving' models (Bitcoin style) vs. 'Linear Decay' vs. 'Dynamic Minting'.]

Traditional cryptocurrencies often rely on rigid, time-based emission schedules, such as Bitcoin's "Halving" model, which reduces supply regardless of demand or network health. However, recent infrastructure protocols have moved toward "Linear Decay" or "State-Dependent" issuance. Most notably, Filecoin's concept of "Effective Network Time" advances issuance only as physical capacity is added to the network \cite{Bernardineli2022}. This creates a feedback mechanism where the protocol slows its inflation during periods of stagnation, preserving value for future growth—a critical design pattern for capital-intensive DePIN projects.

\subsubsection{The Burn-and-Mint Equilibrium (BME)}
[TODO: Detailed definition of BME. How Dual-Token systems (Token + Data Credits) separate volatility from utility cost.]
This structure aligns with the "Web3 Sustainability Loop" proposed by Trent McConaghy (Ocean Protocol), which posits that sustainable data economies requires a closed-loop value flow where network revenue continuously offsets token emissions \cite{McConaghy2020}.

\subsubsection{Work Verification Mechanisms}
[TODO: Define how 'Proof of Physical Work' is validated on-chain (e.g., Proof of Coverage, Proof of Mapping).]
Following the taxonomy of token incentives defined by Lisa JY Tan (Economics Design), Onocoy's mechanism relies on "outcome-based" rewards (verified data) rather than "action-based" rewards, distinguishing it from simpler staking protocols \cite{Tan2020}.

\subsection{Monetization and Demand Regimes}
\label{subsec:demand_regimes}
\subsubsection{The Value Capture Funnel}
[TODO: Map the flow of Fiat -> Buy Pressure -> Burn -> Supply Reduction.]

\subsubsection{Demand Cyclicality: Enterprise vs. Consumer}
[TODO: Contrast steady B2B demand (Geodnet) vs. cyclical consumer demand (Helium).]

\subsection{Detailed Analysis of the ONO Mechanism}
\label{subsec:ono_mechanism}
[TODO: Deep dive into Onocoy's specific 'Capped Supply' architecture vs. the BME industry standard defined above.]

\subsection{Conclusion}
[TODO: Summarize the theoretical vulnerabilities identified in these models, setting the stage for the specific Empirical Stress Analysis in Section 6.]

\newpage

% SCOPE CONTRACT: PERSON A (Empirical Case)
\section{Empirical Case: Onocoy}
\label{sec:onocoy_case}

\textit{[INSTRUCTIONS FOR PERSON A: Please fill this section with descriptive facts. Do not analyze the token price or success yet. Focus on the business and the hardware.]}

\subsection{System Overview}
\begin{itemize}
    \item \textbf{GNSS \& RTK Technology:} Explain what Real-Time Kinematics (RTK) is and why it improves GPS accuracy (centimeter-level precision).
    \item \textbf{The Physical Network:} Describe the density of stations required (e.g., 20km baseline) and why a distributed network makes sense versus a centralized one.
\end{itemize}

\subsection{Network Participants}
\begin{itemize}
    \item \textbf{Miners (Reference Stations):} Define the hardware requirements. Cost of entry? (e.g., \$1,000+). Install requirements? (Roof access, clear sky view).
    \item \textbf{Users (Rovers):} Who buys the data? (Surveyors, drones, agriculture machines). How do they access it? (NTRIP caster).
\end{itemize}

\subsection{Tokenomic Mechanics (Descriptive)}
\begin{itemize}
    \item \textbf{The ONO Token:} Describe the emission schedule (halving? decay?).
    \item \textbf{Burn/Sink Mechanism:} How does the system capture value? (e.g., Data Credits or burn-and-mint).
    \item \textbf{Staking Rules:} Are there staking requirements for miners to participate?
\end{itemize}

\newpage

% SCOPE CONTRACT: PERSON C (Methodology)
% Content moved from Person A's file

\section{Methodology: Simulation-Based Stress Testing}
\label{sec:methodology}

\subsection{Purpose and Non-Goals}
This simulation framework examines how tokenomic mechanisms in Decentralized Physical Infrastructure Networks (DePIN) behave when conditions deteriorate \cite{Messari2024}. Our aim is not to predict future outcomes or identify winning designs, but to compare how different incentive structures respond when exposed to the same adverse scenarios. The emphasis is therefore on \emph{relative behavior} under stress, rather than on absolute performance or forecasting accuracy \cite{Sollaci2004}.

We use simulation because DePIN systems combine several features that are difficult to study analytically. Infrastructure providers are heterogeneous, incentives operate through protocol rules rather than direct contracts, and external conditions such as demand or market sentiment can change abruptly \cite{Ho2022}. These elements interact in non-linear ways, especially once feedback effects emerge. A simulation-based approach makes it possible to examine these interactions directly, while keeping assumptions explicit and inspectable \cite{SantaFeInstitute}.

To avoid over-interpretation, a number of non-goals are defined upfront. The model does not produce token price forecasts, valuation estimates, or projections of market capitalization \cite{CFA2018}. Price dynamics are included only as internal signals that reflect supply, demand, and liquidity assumptions. Likewise, the results are not treated as causal claims about real-world networks. All observed patterns are conditional on the modeled assumptions and should be understood as indicative rather than predictive.

The framework does not identify optimal parameter settings or prescribe specific tokenomic designs. Instead, it surfaces trade-offs and failure modes visible under stress. Finally, the model abstracts from governance processes, legal constraints, and off-chain behavioral factors that may be important in practice but cannot be represented in a tractable and reproducible way within the scope of this study.

This aligns with standard practice in complex systems research, where controlled comparison under adverse conditions is preferred over predictive modeling in settings characterized by uncertainty, feedback effects, and heterogeneous agents \cite{Morris2019, Braakman2022}.

\subsection{Model Scope and Agent Representation}
The simulation model represents DePIN dynamics through a limited set of interacting components that capture the core economic relationships relevant to infrastructure sustainability. The scope is defined by what can be meaningfully stress-tested, rather than by an attempt to mirror the full complexity of real-world networks \cite{BrouwerBurg2016}.

At the center of the model are infrastructure providers, represented as heterogeneous agents. Providers differ along several dimensions that materially affect their economic outcomes, including operational costs, effective capacity contribution, and hardware tier. This heterogeneity is essential, as DePIN networks do not consist of interchangeable participants but of operators with varying cost structures. Provider behavior is limited to economically motivated decisions, primarily continued participation, exit (churn), or entry, based on profitability thresholds and accumulated losses \cite{Mardikes2025}.

Users are not modeled as individual agents. Instead, service demand is treated as an exogenous input, specified through time-varying demand regimes. While real demand may respond endogenously to price or service quality, introducing such feedback would require assumptions that cannot be validated within the scope of this thesis. Demand is therefore used as a controlled external driver, allowing the analysis to focus on how tokenomic mechanisms respond to stress rather than on how demand itself forms.

The protocol is represented as a rule-based system that governs emissions, reward allocation, burn mechanisms, and price-related dynamics. Protocol behavior is deterministic given a set of parameters and inputs, and does not adapt strategically over time. This abstraction is intentional: the objective is to evaluate the behavior of tokenomic mechanisms as designed, not to model governance interventions.

Within this structure, we maintain a clear distinction between endogenous variables (token supply, provider profitability, churn, incentive solvency) and exogenous variables (demand regimes, macroeconomic conditions, liquidity shocks). This separation allows observed outcomes to be traced back to specific stress inputs and mechanism responses.

\subsection{Inputs, Parameters, and Exogenous vs Endogenous Variables}
The simulation framework operates on a clearly defined set of inputs and parameters that determine how the system evolves over time. These inputs are grouped by their role in the model rather than by their real-world source, in order to make assumptions explicit. A central design principle is that parameters are treated as \emph{ranges and regimes}, not as precise estimates of real-world values.

Exogenous inputs define the external environment. These include demand regimes, macroeconomic conditions, and discrete shock events. Demand is specified as a time series that can follow different patterns, such as steady usage, growth followed by decay, or highly volatile behavior. Macroeconomic conditions are represented through scenario-level modifiers. Liquidity shocks, such as large token unlocks, are introduced as one-time exogenous disturbances. Treating these factors as exogenous allows the same stress conditions to be applied consistently across different tokenomic configurations.

Protocol-level parameters define the incentive mechanisms under evaluation. These include emission limits, burn fractions, reward allocation rules, and treasury handling strategies. Parameters in this category are varied across simulation runs to represent alternative tokenomic designs or policy choices. Importantly, the model does not assume that protocol parameters adapt dynamically in response to outcomes. Any change in emissions or incentive structure is introduced explicitly as part of a scenario, rather than emerging endogenously.

Provider-level parameters describe the economic characteristics of infrastructure operators. These include capital expenditure proxies, ongoing operational costs, and participation thresholds. Provider heterogeneity is introduced by sampling these parameters from predefined distributions. Where practitioner input is used (e.g., typical cost ranges or revenue expectations), it is incorporated as broad intervals influenced by public documentation, not as exact figures.

\subsection{Stress Dimensions and Scenario Design}
Stress scenarios in the simulation are designed to represent adverse conditions that are plausible for DePIN networks and that materially challenge incentive alignment, provider retention, and service continuity. The model focuses on a limited set of stress dimensions that recur across incentive-driven infrastructure systems.

A first category of stress relates to demand conditions. Demand regimes are specified exogenously and include stable demand, growth followed by decay, and highly volatile usage patterns. These regimes reflect the well-documented risk that early network adoption does not translate into sustained utilization \cite{Ho2022}.

A second category relates to external market shocks. Macroeconomic regimes (Bear, Bull, Sideways) modulate the baseline assumptions for token price drift and volatility. Liquidity shocks simulate sudden sell pressure events, testing the system's ability to maintain incentive alignment even in the absence of changes in underlying demand \cite{Gauntlet, ChaosLabs}.

Provider-side stress is introduced through economic viability thresholds. In scenarios where operational costs increase, rewards decline, or token prices fall, providers may experience sustained losses. The model represents this through probabilistic churn mechanisms that activate after consecutive periods of unprofitability.

These stress dimensions can be applied individually or in combination, allowing the analysis to explore compound failure conditions. Prior work on complex systems suggests that the interaction of multiple stressors often produces non-linear outcomes that are not apparent when shocks are considered in isolation. By standardizing the definition and timing of stress across scenarios, the model enables direct comparison of how different tokenomic mechanisms absorb or amplify adverse conditions.

\subsection{Metrics and Computation}
The evaluation of tokenomic robustness in this study relies on a fixed set of metrics defined prior to analysis. Metrics are selected based on their ability to capture incentive sustainability, provider behavior, and system-level stress responses, rather than short-term market performance.

A primary category of metrics captures provider sustainability and retention. Provider retention rates and churn counts are used to assess whether incentive structures are able to maintain infrastructure participation under stress. These measures are particularly important in DePIN systems, where provider exit can lead to persistent losses in service capacity due to physical deployment constraints.

A second category relates to economic viability and incentive solvency. Metrics such as provider profitability, net incentive balance, and revenue-to-emission ratios are used to examine whether rewards remain economically meaningful relative to costs over time. Rather than interpreting these values as precise financial indicators, they are used as \emph{proxies} for incentive health.

System-level performance is captured through capacity and utilization metrics, including total active capacity, demand served, and utilization rates. These measures reflect whether the network continues to deliver service under adverse conditions, independent of token price dynamics.

To assess broader tokenomic dynamics, the model tracks supply-side and flow-based indicators, including net emissions, burn-to-mint ratios, and token velocity proxies derived from transaction turnover. These metrics are not interpreted as direct measures of market efficiency, but as signals of whether token circulation and supply pressures remain aligned with usage-driven demand.

Finally, distributional outcomes are examined using concentration metrics, such as top-participant reward shares or inequality proxies. These measures are included to detect whether stress conditions lead to excessive centralization of rewards or capacity, which may undermine network resilience even when aggregate metrics appear stable.

\subsection{Reproducibility and Validation Strategy}
Reproducibility is treated as a core requirement of the simulation framework. All experiments are defined by an explicit set of parameters, scenario inputs, and random seeds, allowing results to be regenerated and inspected without reliance on undocumented assumptions or interactive tuning.

Stochastic elements are present in several parts of the simulation, including demand variation, provider heterogeneity, churn decisions, and price-related noise. To ensure that observed outcomes are not driven by single-run artifacts, each scenario is evaluated across multiple simulation runs using seeded pseudo-random number generation. Results are then aggregated to examine central tendencies and dispersion rather than relying on individual trajectories.

Validation in this context does not aim to establish empirical accuracy with respect to real-world data. Instead, the model is validated through internal consistency and behavioral plausibility checks. These include baseline runs without stress, where the system is expected to exhibit stable participation and bounded dynamics, as well as controlled perturbations where directional responses can be anticipated based on incentive logic.

Sensitivity analysis is used to further assess robustness. Key parameters (cost ranges, emission limits, and churn thresholds) are varied within predefined intervals to examine whether qualitative outcomes persist under moderate perturbations. Outcomes that remain stable across reasonable parameter variations are treated as more robust signals.

\subsection{Limitations of the Model}
While the simulation framework is designed to support structured stress testing, it necessarily abstracts from many aspects of real-world network behavior. These limitations are not incidental, but a consequence of deliberate design choices made to preserve interpretability and reproducibility.

Several limitations shape the model's scope. Demand is modeled as an exogenous process; although some scenarios allow demand to vary over time, the model does not fully endogenize user behavior or price elasticity. Providers are represented as economically rational agents operating under simplified decision rules, excluding strategic behavior such as long-term speculation or coordinated action.

The price formation mechanism is a reduced-form approximation, combining buy and sell pressure, dilution effects, scarcity signals, and stochastic noise to generate directional price dynamics. This approach is sufficient for comparative stress testing but does not attempt to replicate actual market microstructure. The temporal resolution of the model is limited to discrete time steps, obscuring intra-period dynamics such as short-term volatility spikes.

Finally, empirical calibration is constrained by data availability. Inputs derived from interviews or public documentation are treated as indicative ranges rather than precise measurements. The framework is therefore better suited to identifying relative sensitivities and failure modes than to making absolute performance claims.

Taken together, these limitations imply that the results of the simulation should be interpreted as comparative, conditional insights rather than predictions. The value of the model lies in its ability to expose how different tokenomic mechanisms respond under controlled stress scenarios, not in asserting how any particular network will behave in practice.

\subsection{Adversarial Stress and Future Work}
A critical distinction in this framework is the assumption of ``economically rational but honest'' agents. Real-world DePIN networks, however, face existential risks from adversarial behavior, such as Sybil attacks, GPS spoofing, and strategic governance capture \cite{ResonanceSecurity2024}. While the current model captures economic stress, it does not explicitly promote ``Scenario S5: Verification Failure,'' where malicious actors dilute rewards for honest participants, potentially triggering a ``Lemons Market'' dynamic \cite{EconAgentic2025}.

Future iterations of this framework should integrate adversarial agent types to evaluate the resilience of verification mechanisms (e.g., Proof of Physical Work) against coordinated spoofing. This would allow for a more comprehensive assessment of ``Anti-Fragile'' tokenomics that includes security vectors alongside economic ones \cite{GeodnetResearch}.

\section{Stress Scenario Design}
\label{sec:stress_scenarios}

\subsection{Operational Definition of Stress}
In the context of this thesis, stress is defined operationally as an externally imposed adverse condition that challenges the ability of a DePIN network to maintain functional capacity through its incentive system. Stress is not treated as an outcome, such as price decline or provider exit, but as an input condition applied to the system.

This distinction is critical. By defining stress as an exogenous input, the analysis avoids circular reasoning in which system failure is both the cause and the consequence of stress. Instead, stress scenarios are constructed as controlled perturbations to the environment in which tokenomic mechanisms operate, allowing their responses to be observed and compared.

Stress, as modeled here, targets three core dimensions of network sustainability:
\begin{enumerate}
    \item Provider retention, reflecting the willingness of infrastructure operators to remain active.
    \item Service continuity, reflecting the network's ability to meet demand through available capacity.
    \item Incentive solvency, reflecting the balance between token issuance, usage-driven sinks, and economic viability.
\end{enumerate}

These dimensions align with established interpretations of robustness in complex systems, where resilience is assessed by continued function rather than by avoidance of disturbance \cite{Braakman2022}.

\subsection{Stress Dimensions Implemented}
The simulation framework implements multiple stress dimensions, each corresponding to real-world risk factors commonly observed in crypto-economic systems and infrastructure networks. All stress dimensions are treated as exogenous inputs, consistent with the scope constraints defined in Section \ref{sec:methodology}.

\subsubsection{Macroeconomic Regimes}
Macroeconomic conditions are represented through regime-based modifiers that influence token price drift and volatility. Three regimes are defined:
\begin{itemize}
    \item Bearish, characterized by negative drift and elevated volatility.
    \item Sideways, characterized by neutral drift and moderate volatility.
    \item Bullish, characterized by positive drift and reduced volatility.
\end{itemize}

These regimes abstract broader market sentiment and liquidity conditions without attempting to model specific macroeconomic variables. Their purpose is to expose tokenomic mechanisms to varying background conditions rather than to replicate real-world cycles.

\subsubsection{Demand Regimes}
Demand for network services is modeled as a stochastic time series with configurable structure. Four demand regimes are implemented:
\begin{itemize}
    \item Stable demand, with low variance around a constant mean.
    \item Growth demand, exhibiting sustained upward trends.
    \item Volatile demand, characterized by high variance without long-term trend.
    \item High-to-decay demand, representing an initial surge followed by gradual decline.
\end{itemize}

These regimes reflect common adoption patterns observed in infrastructure services, including early hype cycles and post-deployment normalization. Demand remains exogenous by design, enabling isolation of provider-side incentive responses.

\subsubsection{Liquidity Shocks}
Liquidity stress is introduced through discrete token unlock events combined with finite market depth. These events simulate situations in which large token holders sell a portion of their holdings into an automated market maker pool, producing abrupt price shocks.

Liquidity shocks are parameterized by:
\begin{itemize}
    \item Unlock timing
    \item Proportion of circulating supply unlocked
    \item Available liquidity depth
\end{itemize}

The objective is not to model investor behavior but to evaluate how tokenomic mechanisms respond to sudden price dislocations.

\subsubsection{Provider Economics Stress}
Provider-side economic stress is modeled through increases in operational costs, competitive yield opportunities, and reduced profitability thresholds. These inputs reflect real-world pressures faced by infrastructure operators, such as rising energy prices or alternative revenue opportunities.

In addition, the model distinguishes between provider types (e.g., higher-cost professional installations versus lower-cost basic setups), allowing differential sensitivity to stress.

\subsubsection{Tokenomics Parameter Stress}
Finally, stress is applied directly to tokenomic parameters, including emission caps, burn fractions, and initial supply conditions. This allows evaluation of how sensitive system behavior is to design choices, rather than to external shocks alone.

\subsection{Scenario Combination Rules}
Stress dimensions are combined according to clearly defined rules. Parameter-based stresses (e.g., macro regime, demand regime, provider costs) may be combined freely, allowing compound stress scenarios to be evaluated. In contrast, scenario-specific behaviors (such as saturation or utility-focused modes) are treated as mutually exclusive to preserve interpretability.

This design choice reflects a trade-off between realism and experimental control. While real-world systems may experience overlapping structural transitions, isolating scenario-specific behaviors reduces confounding effects and supports clearer comparative analysis.

\subsection{Relevance of Stress Scenarios}
The stress scenarios implemented in this thesis are not exhaustive representations of all possible risks faced by DePIN networks. Rather, they are selected to capture representative failure pressures commonly cited in practitioner and academic literature, including subsidy dependency, liquidity fragility, and provider churn \cite{Messari2024, Ho2022}.

By standardizing these scenarios, the framework enables consistent comparison across tokenomic mechanisms and protocol profiles.

\section{Evaluation Metrics}
\label{sec:evaluation_metrics}

\subsection{Principles for Metric Selection}
Evaluating the robustness of tokenomic mechanisms requires metrics that are both interpretable and aligned with the functional objectives of DePIN networks. Three principles guide metric selection in this thesis: metrics must be comparative, enabling evaluation of relative performance across mechanisms and scenarios rather than absolute success or failure; they must be mechanism-relevant, reflecting provider incentives, service delivery, or economic sustainability rather than speculative market outcomes; and they must be transparent, with clearly defined computation and interpretation.

These principles align with established guidance for simulation-based evaluation, where metrics should illuminate system behavior without overstating precision \cite{Morris2019}.

\subsection{Core Sustainability Metrics}
The following core metrics are used to evaluate system behavior under stress.

\subsubsection{Provider Retention and Churn}
Provider retention is measured as the proportion of active providers remaining over time, while churn captures the rate and magnitude of provider exit events. These metrics directly reflect the stability of the infrastructure layer and are central to DePIN sustainability.

Retention is interpreted directionally: higher retention under comparable stress indicates greater robustness, but does not imply optimal or permanent stability.

\subsubsection{Capacity Utilization and Service Continuity}
Service continuity is assessed through capacity utilization and demand satisfaction rates. Capacity utilization measures the extent to which deployed infrastructure is effectively used, while demand satisfaction captures the proportion of demand that can be served given available capacity.

Together, these metrics indicate whether incentive mechanisms support not only provider participation but also functional service delivery.

\subsubsection{Incentive Solvency (Burn-to-Mint Ratio)}
Incentive solvency is proxied by the ratio of value burned through usage-driven sinks to value emitted through token issuance. A ratio approaching or exceeding unity suggests reduced reliance on subsidies, while persistent divergence indicates ongoing subsidy dependency.

This metric abstracts complex monetary dynamics into a single comparative signal and is interpreted cautiously as an indicator rather than a definitive threshold.

\subsubsection{Net Emissions and Inflation Dynamics}
Net emissions capture the balance between token issuance and destruction over time. While absolute inflation rates are not used as targets, changes in net emissions under stress reveal how mechanisms respond to shifts in demand and participation.

\subsubsection{Token Velocity (Proxy)}
Token velocity is estimated using transaction turnover as a proxy. While this does not capture all aspects of monetary velocity, it provides insight into whether tokens circulate primarily as a medium of exchange or accumulate as idle balances.

Velocity is interpreted comparatively across scenarios and mechanisms, not as an absolute indicator of economic health.

\subsubsection{Volatility Proxy}
Price volatility is measured using relative dispersion metrics derived from simulated price series. Given the simplified market representation, volatility is treated as a stress signal affecting provider incentives rather than as a market performance indicator.

\subsection{Derived Summary Indicators}
Several derived indicators are computed to support synthesis and interpretation.

\subsubsection{Death Spiral Probability}
Death spiral probability is defined operationally as the frequency with which simulations exhibit concurrent declines in price, provider count, and service capacity beyond defined thresholds. This indicator captures compounded failure dynamics without asserting inevitability.

\subsubsection{Network Revenue and Provider Profitability}
Aggregate network revenue and average provider profitability are tracked to contextualize incentive outcomes. These metrics help distinguish between retention driven by genuine economic viability and retention sustained by temporary subsidies.

\subsection{Interpretation and Limitations of Metrics}
All metrics used in this thesis are subject to limitations arising from model assumptions and abstraction. In particular, simplified price formation and exogenous demand constrain the interpretation of monetary indicators. Accordingly, metrics are used to compare relative robustness, not to establish optimal designs or real-world predictions.

This cautious interpretation aligns with best practices in simulation-based research and supports the thesis's evaluative rather than predictive orientation \cite{Braakman2022}.

\newpage

% SCOPE CONTRACT: PERSON C (Modeling & Results)
% MUST:
% - Describe the simulation methodology (System Dynamics/Agent-Based)
% - Define specific scenarios (Base case, Bull, Bear, Stress)
% - Define specific KPIs and success metrics
% - Present quantitative results, charts, and sensitivity analysis
% MUST NOT:
% - Re-explain the tokenomics mechanics (Person B)
% - Re-define foundational stress testing concepts (Person A)



\section{Simulation Results}
\label{sec:simulation_results}

\subsection{Baseline Behavior (Sanity Check)}
\label{baseline-behavior-sanity-check}

This section presents the results of the stress-testing simulations
described in Chapter 5. The purpose is to report observed outcomes under
standardized adverse scenarios, not to evaluate their desirability or to
draw normative conclusions. Interpretation and implications are deferred
to Chapter 7.

Results are organized around comparative stress responses of
tokenomic mechanisms rather than absolute performance levels. Each
experiment evaluates how a given protocol configuration behaves relative
to others when exposed to the same stress inputs. This structure
reflects the central aim of the thesis: to assess directional
robustness and failure sensitivity, not to rank protocols by success or
forecast real-world outcomes.

The results are grouped into four thematic blocks:

\begin{enumerate}
\def\labelenumi{\arabic{enumi}.}
\item
 Baseline Behavior Under Neutral Conditions\\ Establishes reference trajectories for price, supply, provider
 participation, and incentive balance in the absence of external
 stress. These runs serve solely as comparison anchors and are not
 interpreted as equilibrium states.
\item
 Response to Demand-Side Stress\\ Examines how tokenomic mechanisms react to sudden demand
 contraction, demand volatility, and high-to-decay demand regimes.
 Outcomes are evaluated in terms of provider retention, utilization,
 and emission efficiency.
\item
 Response to Supply- and Liquidity-Side Stress\\ Analyzes the effects of emission pressure, burn intensity, and
 liquidity shocks such as investor unlock events. Focus is placed on
 dilution dynamics, price drawdowns, and churn amplification.
\item
 Compound Stress and Failure Thresholds\\ Evaluates system behavior under combined stressors, including
 bearish macro regimes, adverse provider economics, and competitive
 yield pressure. These scenarios are used to identify tipping points
 where incentive mechanisms cease to function as intended.
\end{enumerate}

For each block, results are presented using a consistent set of metrics
defined in Section 5.3. Where stochastic variation is present, outcomes
are summarized using distributional statistics (median, interquartile
range, and tail behavior) across multiple simulation runs. Individual
time series are shown only where they illustrate structurally meaningful
patterns.

All figures follow the same conventions:

\begin{itemize}
\item
 identical time horizons,
\item
 shared axes where comparison is intended,
\item
 and standardized scenario labels.
\end{itemize}

Unless explicitly stated otherwise, all simulations use identical demand
processes, macro regimes, and random seeds across protocol profiles to
ensure comparability.

It is important to emphasize that no result in this chapter
should be read as a performance claim about any real-world network.
Observed behaviors reflect the interaction between modeled mechanisms
and imposed stress conditions under the assumptions documented in
Chapter 5. The value of these results lies in their comparative
consistency and in the patterns they reveal across scenarios, not in
their absolute magnitudes.

\subsubsection{Experimental Setup}\label{experimental-setup}

Baseline simulations are conducted over a fixed horizon of 52 weeks
using a weekly timestep. For each protocol profile, 100 Monte Carlo
simulations are executed with a deterministic master seed strategy to
ensure reproducibility. All simulations use the agent-based execution
model with identical neutrality constraints applied across profiles.

Key baseline conditions include a consistent demand regime with zero
demand volatility, a sideways macroeconomic regime, and the explicit
disabling of scenario logic, investor unlocks, competitor yield effects,
and growth shocks. These constraints ensure that observed dynamics arise
solely from the interaction between protocol-specific tokenomic
mechanisms and provider-level economics, rather than from exogenous
stress factors.

The baseline profiles evaluated in this section include Onocoy (ONO) and
a set of comparator DePIN protocols representing a range of
infrastructure types and incentive designs: Helium, Render, Filecoin,
Akash, Hivemapper, DIMO, Grass, io.net, Nosana, and Geodnet.

\subsubsection{Baseline Trajectories}\label{baseline-trajectories}

Across all profiles, baseline simulations generate stable and smooth
trajectories for core state variables such as token supply, emissions,
burns, provider count, aggregate capacity, and utilization. In the
absence of shocks or volatility, these trajectories reflect the
steady-state behavior implied by each protocol's emission logic, reward
allocation rules, and provider cost structure.

Token price series under baseline conditions exhibit limited dispersion
in early periods for some profiles, with the interquartile range
collapsing to a single value at certain timesteps. This behavior is a
direct consequence of strict neutrality assumptions, including zero
demand noise and the absence of stochastic macro or liquidity effects.
Under these conditions, price evolution becomes highly constrained by
deterministic components of the model.

By contrast, provider-level economic metrics such as average provider
profit display visible dispersion across simulations, even under neutral
conditions. For example, the interquartile range of average provider
profit for the ONO profile diverges meaningfully from the median within
the first few weeks, reflecting heterogeneity in provider costs,
capacity, and churn dynamics. This confirms that the Monte Carlo
structure is functioning as intended and that stochastic variation
persists at the agent level even when aggregate demand and macro inputs
are fixed.

\subsubsection{Provider Economics and
Retention}\label{provider-economics-and-retention}

Baseline retention metrics indicate that provider populations remain
largely stable over the 52-week horizon for profiles that begin with a
non-zero provider count. Where a profile's initial provider count is
zero, retention metrics are reported as undefined rather than imputed,
in order to avoid misleading normalization.

Retention is reported in two complementary forms: absolute retention,
defined as the ratio of active providers relative to the initial
provider count, and week-to-week retention rate derived from observed
churn events. Under baseline conditions, churn events are driven solely
by provider-level profitability thresholds and cost structures, without
amplification from price shocks or demand collapses.

These baseline retention trajectories serve as a reference point for
later stress scenarios, where deviations from baseline behavior can be
attributed to specific adverse conditions rather than to underlying
model instability.

\subsubsection{Interpretation Boundaries of Baseline
Results}\label{interpretation-boundaries-of-baseline-results}

It is important to emphasize that baseline results are descriptive
rather than evaluative. Observed trends in price, profitability, or
provider dynamics under neutral conditions should not be interpreted as
indicators of real-world performance or sustainability. Instead, they
establish a controlled reference environment in which the internal
mechanics of each tokenomic system operate without external stress.

The baseline therefore functions as a calibration layer for subsequent
analysis. In later sections, stressed scenarios are evaluated relative
to these baseline trajectories, allowing deviations in retention,
incentive solvency, and service continuity to be attributed to specific
stress mechanisms rather than to baseline model behavior.

\subsection{Stress Scenarios: Definition and
Execution}\label{stress-scenarios-definition-and-execution}

This section defines the adverse conditions under which DePIN tokenomic
mechanisms are evaluated in subsequent analyses. Each stress scenario is
designed to isolate a specific class of risk commonly encountered by
real-world DePIN deployments. Scenarios are implemented in a controlled
and standardized manner to ensure comparability across protocol profiles
and to enable attribution of observed effects to specific stress
mechanisms rather than to baseline dynamics.

All stress scenarios build directly on the neutral baseline
configuration described in Section 6.2. Unless explicitly stated
otherwise, only one scenario-specific stress mechanism is activated at a
time. All other parameters remain identical to the baseline
configuration.

\subsubsection{Rationale for
Scenario-Based Stress
Testing}\label{rationale-for-scenario-based-stress-testing}

DePIN networks operate at the intersection of physical infrastructure,
economic incentives, and market volatility. As a result, system failure
is rarely caused by a single factor in isolation. However, evaluating
compound shocks without first understanding individual stress responses
risks obscuring causal pathways.

Scenario-based stress testing is therefore employed to examine
directional robustness under clearly defined adverse conditions. This
approach aligns with established practices in simulation-based
evaluation of complex systems, where standardized shocks are used to
test mechanism sensitivity and failure modes before compound
interactions are introduced.

The scenarios defined below reflect empirically observed risks in DePIN
and adjacent crypto-economic systems, including demand contraction,
liquidity shocks, competitive pressure, and provider-side economic
stress.

\subsubsection{Scenario S1: Demand
Contraction}\label{scenario-s1-demand-contraction}

The demand contraction scenario models a sustained reduction in service
demand relative to baseline conditions. This scenario reflects
situations such as market downturns, loss of enterprise clients,
regulatory friction, or delayed adoption of the underlying service.

Implementation:

\begin{itemize}
\item
 Demand regime is modified from consistent baseline demand to a
 declining or suppressed demand trajectory.
\item
 No changes are made to emission schedules, provider costs, or
 liquidity conditions.
\item
 Demand volatility may remain low to isolate level effects rather than
 noise-induced effects.
\end{itemize}

Analytical focus:

\begin{itemize}
\item
 Burn-to-emission dynamics under reduced utilization.
\item
 Provider profitability as a function of fixed operating costs.
\item
 Early signals of incentive insolvency without price shocks.
\end{itemize}

\subsubsection{Scenario S2: Liquidity
Shock}\label{scenario-s2-liquidity-shock}

The liquidity shock scenario simulates a sudden increase in sell
pressure resulting from token unlocks or large holder exits. This
scenario reflects common events in crypto markets such as investor
vesting cliffs or coordinated exits during market stress.

Implementation:

\begin{itemize}
\item
 A discrete investor unlock event is introduced at a predefined
 simulation week.
\item
 A fixed percentage of circulating supply is sold into available
 liquidity.
\item
 All other parameters remain identical to baseline conditions.
\end{itemize}

Analytical focus:

\begin{itemize}
\item
 Price sensitivity to sell pressure.
\item
 Secondary effects on provider revenue denominated in token value.
\item
 Churn amplification driven by price-mediated profitability thresholds.
\end{itemize}

\subsubsection{Scenario S3: Competitive
Yield
Pressure}\label{scenario-s3-competitive-yield-pressure}

This scenario introduces external competition for infrastructure
providers by simulating an alternative yield opportunity. It reflects
situations where providers can redeploy capital or hardware to competing
networks offering higher short-term returns.

Implementation:

\begin{itemize}
\item
 A competitor yield parameter is activated, increasing the opportunity
 cost of participation.
\item
 Provider churn probability is adjusted based on relative
 profitability.
\item
 No direct changes are made to demand or token supply.
\end{itemize}

Analytical focus:

\begin{itemize}
\item
 Elasticity of provider participation.
\item
 Sensitivity of retention to marginal profit compression.
\item
 Differentiation between high-commitment and low-commitment providers.
\end{itemize}

\subsubsection{Scenario S4: Provider Cost
Inflation}\label{scenario-s4-provider-cost-inflation}

The provider cost inflation scenario models increases in operating
expenses, such as energy prices, maintenance costs, or regulatory
compliance burdens. This scenario isolates supply-side stress
independent of demand or market conditions.

Implementation:

\begin{itemize}
\item
 Provider operating costs are increased uniformly or by tier.
\item
 Tokenomic parameters remain unchanged.
\item
 Demand and price regimes follow baseline assumptions.
\end{itemize}

Analytical focus:

\begin{itemize}
\item
 Margin compression at the provider level.
\item
 Differential effects across provider types.
\item
 Early indicators of infrastructure attrition driven by cost pressure.
\end{itemize}

\subsubsection{Execution and
Comparability}\label{execution-and-comparability}

All stress scenarios are executed using the same Monte Carlo structure
as the baseline, with identical simulation counts, time horizons, and
seed strategies. This ensures that observed differences between
scenarios and baseline trajectories are attributable to
scenario-specific inputs rather than to stochastic artifacts or
configuration drift.

Scenario outputs are recorded using the same metric set and data
structures as baseline simulations. This enables direct comparison
across scenarios and supports consistent visualization and statistical
summarization in subsequent sections.

\subsection{Stress Scenario Results
(Comparative
Outcomes)}\label{stress-scenario-results-comparative-outcomes}

This section reports the observed outcomes of the stress scenarios
defined in Section 6.3. Results are presented as deviations from the
neutral baseline trajectories established in Section 6.2. The focus is
on \emph{what changes}, \emph{when changes occur}, and \emph{which
metrics are affected}, without attributing causality or normative
judgment. Interpretation of these outcomes is deferred to subsequent
sections.

All results are aggregated across 100 Monte Carlo simulations per
protocol profile. Unless stated otherwise, reported trajectories refer
to median values, with interquartile ranges (p25--p75) used to
illustrate dispersion where relevant.

\subsubsection{Demand Contraction
Scenario}\label{demand-contraction-scenario}

Under demand contraction, all protocol profiles exhibit a measurable
decline in utilization, followed by a widening divergence between token
emissions and realized burns. This effect manifests earliest in metrics
directly tied to service usage, including demand served, utilization
rate, and burn-derived revenue.

Across profiles, reduced utilization leads to lower average provider
revenue while emissions continue according to protocol-specific
schedules. As a result, the burn-to-emission ratio declines relative to
baseline, particularly in protocols with fixed or weakly demand-linked
emission logic. Provider counts remain initially stable but begin to
diverge from baseline trajectories as profitability thresholds are
breached for marginal providers.

Dispersion across simulations increases over time for provider-level
metrics such as average provider profit and churn counts, indicating
sensitivity to heterogeneous cost structures even under identical demand
trajectories.

\subsubsection{Liquidity Shock
Scenario}\label{liquidity-shock-scenario}

The liquidity shock scenario produces immediate and discrete deviations
in token price trajectories at the point of the unlock event. The
magnitude of the price response varies significantly across protocol
profiles, reflecting differences in circulating supply, liquidity depth,
and emission rates.

Following the shock, secondary effects are observed in provider
economics. In profiles where provider rewards are denominated primarily
in the native token, reductions in token price translate into compressed
real revenue, leading to increased churn in subsequent weeks. This
effect is not instantaneous but unfolds with a short lag, corresponding
to reward distribution timing and provider decision thresholds.

Notably, while price volatility spikes sharply around the unlock event,
other metrics such as utilization and capacity remain initially close to
baseline. This temporal decoupling highlights the indirect transmission
of liquidity stress from token markets to infrastructure participation.

\subsubsection{Competitive Yield Pressure
Scenario}\label{competitive-yield-pressure-scenario}

When external yield pressure is introduced, provider participation
becomes more elastic across all evaluated profiles. The most pronounced
effects are observed in provider churn metrics rather than in price or
utilization measures.

Provider exit accelerates in profiles where baseline profitability
margins are narrow, while protocols with higher average provider surplus
show greater resistance. The dispersion of outcomes across simulations
increases markedly, indicating that small differences in provider
economics can lead to divergent participation trajectories under
competitive pressure.

Unlike liquidity shocks, competitive yield pressure does not produce
abrupt discontinuities in token price or supply. Instead, its effects
accumulate gradually, primarily through sustained differences in
provider retention relative to baseline.

\subsubsection{Provider Cost Inflation
Scenario}\label{provider-cost-inflation-scenario}

Provider cost inflation directly impacts average provider profit across
all protocol profiles. Increases in operating costs reduce margins
uniformly, but downstream effects vary depending on reward structure and
provider heterogeneity.

In several profiles, median provider profit crosses below zero earlier
than under other stress scenarios, leading to persistent negative
profitability even in the absence of price shocks or demand contraction.
Churn metrics respond accordingly, with elevated exit rates observed
once cost thresholds are exceeded.

Importantly, capacity and utilization do not decline immediately in this
scenario. Instead, reductions in active providers precede observable
changes in service capacity, indicating a lag between economic stress
and infrastructure degradation.

\subsubsection{Cross-Scenario
Comparison}\label{cross-scenario-comparison}

Comparing across stress scenarios reveals distinct temporal and
metric-specific signatures:

\begin{itemize}
\item
 Demand contraction primarily affects burn-linked metrics and
 utilization.
\item
 Liquidity shocks propagate rapidly through price metrics before
 affecting provider participation.
\item
 Competitive yield pressure manifests as gradual erosion of provider
 counts.
\item
 Cost inflation exerts sustained pressure on provider profitability
 with delayed capacity effects.
\end{itemize}

Across all scenarios, deviations from baseline trajectories are most
pronounced in provider-level metrics, while aggregate supply and
emission variables tend to remain closer to baseline unless explicitly
modified by the scenario.

These observed patterns form the empirical basis for identifying
recurring failure modes in DePIN tokenomic systems, which are formalized
in the following section.

\subsection{Failure Modes Observed
(Operational
Definitions)}\label{failure-modes-observed-operational-definitions}

Table \ref{tab:failure_modes} provides a diagnostic matrix for identifying the five recurring failure modes formalized in this section.

\begin{table}[ht]
    \centering
    \small
    \renewcommand{\arraystretch}{1.3} 
    \begin{tabularx}{\textwidth}{@{} 
        >{\raggedright\arraybackslash\hsize=0.85\hsize}X 
        >{\raggedright\arraybackslash\hsize=1.25\hsize}X 
        >{\raggedright\arraybackslash\hsize=0.9\hsize}X 
    @{}}
        \toprule
        \textbf{Failure Mode} & \textbf{Operational Definition} & \textbf{Precursor Metric} \\
        \midrule
        \textbf{Reward--Demand Decoupling} & Emissions persist despite declining usage. & Declining burn-to-emission ratio. \\
        \addlinespace
        \textbf{Profitability-Induced Churn} & Exit driven by sustained negative margins (not price). & Avg provider profit $<0$ for $N$ weeks. \\
        \addlinespace
        \textbf{Liquidity-Driven Compression} & Price crash reduces real reward value, causing delayed churn. & Price shock $\to$ lagged exit. \\
        \addlinespace
        \textbf{Elastic Provider Exit} & High sensitivity to yield relative to competitors. & Elevated churn despite stable internal metrics. \\
        \addlinespace
        \textbf{Latent Capacity Degradation} & Service capacity drops long after economic stress begins. & Provider count drops before capacity drops. \\
        \bottomrule
    \end{tabularx}
    \caption{Operational Failure Modes: Diagnostic Matrix}
    \label{tab:failure_modes}
\end{table}



This section formalizes the recurring failure modes observed across
stress scenarios by defining them in operational, measurable terms.
Failure modes are not treated as binary outcomes or protocol-level
judgments. Instead, they are defined as \emph{patterns of metric
behavior} that emerge consistently under specific adverse conditions and
indicate increasing fragility in the incentive system.

Each failure mode is grounded in observable deviations from baseline
trajectories and is specified using explicit metric signatures. These
definitions provide a common vocabulary for comparing DePIN tokenomic
systems and serve as the analytical bridge between empirical results and
subsequent implications.

\subsubsection{Reward--Demand
Decoupling}\label{rewarddemand-decoupling}

Definition:\\ Reward--demand decoupling occurs when token emissions and provider
rewards remain elevated relative to realized service demand, resulting
in a sustained divergence between minting and burn-linked revenue.

Operational indicators:

\begin{itemize}
\item
 Persistent decline in utilization relative to baseline.
\item
 Decreasing burn-to-emission ratio over time.
\item
 Stable or increasing emissions despite reduced demand served.
\item
 Compression of average provider profit without immediate price
 collapse.
\end{itemize}

Observed context:\\ This pattern is most clearly observed under demand contraction
scenarios, where reduced usage does not immediately propagate into lower
emissions. The decoupling persists until provider profitability
thresholds are crossed, at which point secondary effects such as churn
begin to emerge.

\subsubsection{Profitability-Induced
Provider
Churn}\label{profitability-induced-provider-churn}

Definition:\\ Profitability-induced provider churn refers to accelerated provider
exit driven by sustained negative or marginal profitability, independent
of abrupt market shocks.

Operational indicators:

\begin{itemize}
\item
 Average provider profit crossing below zero for multiple consecutive
 periods.
\item
 Gradual increase in churn count relative to baseline.
\item
 Divergence between provider count trajectories and baseline without
 corresponding demand or price shocks.
\item
 Increasing dispersion in provider-level outcomes across simulations.
\end{itemize}

Observed context:\\ This failure mode is prominent in provider cost inflation and
competitive yield pressure scenarios, where economic pressure
accumulates slowly rather than arriving as a discrete event.

\subsubsection{Liquidity-Driven Incentive
Compression}\label{liquidity-driven-incentive-compression}

Definition:\\ Liquidity-driven incentive compression occurs when abrupt price
declines reduce the real value of token-denominated rewards, leading to
downstream effects on provider participation despite unchanged nominal
reward structures.

Operational indicators:

\begin{itemize}
\item
 Sharp, localized price drawdowns coinciding with liquidity events.
\item
 Lagged decline in average provider profit following price shocks.
\item
 Increased churn after reward distribution cycles.
\item
 Temporary decoupling between utilization/capacity metrics and provider
 participation.
\end{itemize}

Observed context:\\ This pattern emerges most clearly in liquidity shock scenarios, where
price volatility precedes changes in provider behavior. The delayed
response highlights the indirect transmission of market stress into
infrastructure-level outcomes.

\subsubsection{Elastic Provider Exit Under
External Yield
Pressure}\label{elastic-provider-exit-under-external-yield-pressure}

Definition:\\ Elastic provider exit describes heightened sensitivity of provider
participation to marginal changes in relative yield, resulting in
sustained attrition without dramatic price or demand movements.

Operational indicators:

\begin{itemize}
\item
 Elevated churn rates under competitive yield pressure relative to
 baseline.
\item
 Minimal contemporaneous changes in token price or utilization.
\item
 Increasing dispersion in provider retention outcomes.
\item
 Gradual erosion of provider count rather than abrupt collapse.
\end{itemize}

Observed context:\\ This failure mode is observed when alternative opportunities are
introduced, even in the absence of internal protocol stress. It reflects
the substitutability of provider capital and labor across competing
networks.

\subsubsection{Latent Capacity
Degradation}\label{latent-capacity-degradation}

Definition:\\ Latent capacity degradation refers to delayed reductions in service
capacity that follow earlier provider exits, resulting in a temporal gap
between economic stress and observable service degradation.

Operational indicators:

\begin{itemize}
\item
 Provider count declines preceding reductions in aggregate capacity.
\item
 Stable utilization metrics masking declining redundancy.
\item
 Capacity drops occurring several periods after profitability-induced
 churn.
\item
 Reduced resilience to subsequent demand or liquidity shocks.
\end{itemize}

Observed context:\\This pattern appears across multiple stress scenarios and highlights
the non-instantaneous relationship between provider economics and
service-level outcomes in infrastructure-based systems.

\subsubsection{Role of Failure Mode
Definitions}\label{role-of-failure-mode-definitions}

These failure modes do not imply protocol failure in an absolute sense.
Rather, they represent \emph{structural stress signatures} that can
accumulate, interact, and amplify under adverse conditions. By defining
failure modes operationally, the analysis enables systematic comparison
across DePIN designs and provides a foundation for evaluating mitigation
strategies in subsequent sections.

\subsection{Implications for Builders (Design Recommendations)}\label{implications-7.1-implications-for-builders-design-recommendations}

The stress-testing results and observed failure modes presented in
Section 6 highlight several recurring design tensions inherent to DePIN
tokenomic systems. These implications are not prescriptive blueprints,
but design considerations derived from empirically observed behavior
under adverse conditions. They are intended to inform protocol designers
about where fragility tends to emerge and which mechanisms warrant
particular attention during design and iteration.

\subsubsection{Align Emissions More Tightly with Realized
Demand}\label{align-emissions-more-tightly-with-realized-demand}

One of the most consistently observed failure modes is reward--demand
decoupling, in which emissions and rewards persist despite declining
utilization. This suggests that fixed or weakly demand-sensitive
emission schedules can amplify stress during periods of demand
contraction.

Design implication:\\
Builders should prioritize emission mechanisms that respond meaningfully
to realized service usage rather than to participation alone. Emission
logic that adapts to utilization or burn-linked signals can reduce the
accumulation of excess supply during downturns and delay the onset of
profitability-induced churn.

This does not imply fully reactive or highly volatile emissions, but
rather bounded responsiveness that maintains incentive alignment when
demand deviates from growth assumptions.

\subsubsection{Design for Provider Margin
Resilience, Not Peak
Returns}\label{design-for-provider-margin-resilience-not-peak-returns}

Across multiple scenarios, provider exit is driven less by short-term
volatility and more by sustained margin compression. Once average
provider profitability crosses below zero for extended periods, churn
accelerates even in the absence of dramatic market shocks.

Design implication:\\
Tokenomic systems should be evaluated against downside margin scenarios
rather than upside reward narratives. Builders should stress-test
provider economics under elevated costs, reduced demand, and compressed
prices to ensure that a meaningful subset of providers remains
marginally profitable under adverse conditions.

This may involve conservative reward baselines, differentiated
incentives for higher-commitment providers, or mechanisms that reduce
cost exposure during unfavorable regimes.

\subsubsection{Treat Liquidity Events as
Infrastructure Risks, Not Just Market
Events}\label{treat-liquidity-events-as-infrastructure-risks-not-just-market-events}

Liquidity shocks primarily manifest as price events, but their
downstream effects on provider participation reveal that they function
as infrastructure risks. The lagged response between price declines and
provider churn suggests that market stress propagates through reward
valuation rather than immediate behavioral reaction.

Design implication:\\
Builders should explicitly account for liquidity events when designing
reward distribution and treasury strategies. Mechanisms that smooth
reward value, delay exposure to price shocks, or introduce buffers
between market volatility and provider compensation can reduce the
indirect transmission of liquidity stress into infrastructure attrition.

Ignoring liquidity dynamics as ``external'' risks underestimates their
impact on long-term network stability.

\subsubsection{Assume Provider Capital Is
Mobile}\label{assume-provider-capital-is-mobile}

Competitive yield pressure demonstrates that provider participation is
often elastic, even when demand and protocol fundamentals remain
unchanged. Providers compare returns across networks, and small relative
differences can lead to sustained attrition over time.

Design implication:\\
DePIN tokenomics should be designed with the assumption that provider
capital and labor are mobile. Retention should not rely solely on
short-term reward competitiveness but should incorporate mechanisms that
reward persistence, commitment, or sunk-cost investments where
appropriate.

This may include tenure-based incentives, differentiated rewards by
hardware tier, or gradual reward vesting that aligns provider time
horizons with network needs.

\subsubsection{Account for Delayed
Infrastructure
Degradation}\label{account-for-delayed-infrastructure-degradation}

Latent capacity degradation highlights a structural characteristic of
DePIN systems: infrastructure quality may appear stable even as economic
stress accumulates. By the time capacity reductions become visible,
recovery may be slow or costly due to physical deployment constraints.

Design implication:\\
Builders should monitor leading indicators of stress---such as declining
provider margins or increasing churn dispersion---rather than relying
solely on utilization or capacity metrics. Early-warning indicators
embedded in protocol analytics can provide time to adjust incentives
before service degradation becomes apparent.

Designing for observability is therefore as important as designing for
incentives.

\subsubsection{Positioning Note
(important)}\label{positioning-note-important}

These implications are intentionally framed as \emph{design
considerations}, not as claims of optimality. They reflect tendencies
observed across modeled systems under standardized stress conditions and
should be interpreted as guidance for further testing and refinement
rather than as universal rules.

\subsection{Implications for Onocoy
(Actionable
Suggestions)}\label{implications-for-onocoy-actionable-suggestions}

The failure modes and stress responses observed in this study have
direct relevance for the Onocoy network, given its reliance on
physically deployed GNSS infrastructure and a token-based incentive
system. The following implications translate the general design
considerations outlined in Section 7.1 into Onocoy-specific suggestions.
These are not presented as prescriptions, but as areas where targeted
experimentation and monitoring could improve resilience under adverse
conditions.

\subsubsection{Strengthen the Coupling
Between Usage and
Rewards}\label{strengthen-the-coupling-between-usage-and-rewards}

Baseline and stress simulations indicate that demand-linked metrics are
central to maintaining incentive alignment over time. For Onocoy, where
service usage can be measured with relatively high precision, there is
an opportunity to reinforce the relationship between realized RTK demand
and reward distribution.

Actionable suggestion:\\
Onocoy could explore mechanisms that increase the sensitivity of rewards
to verified service usage, particularly during periods of demand
contraction. This may involve adjusting reward weighting based on
utilization signals or introducing soft constraints that limit emissions
when burn-linked revenue consistently lags behind supply issuance.

Such adjustments would not need to be abrupt or highly reactive; even
gradual modulation could reduce prolonged reward--demand decoupling.

\subsubsection{Monitor Provider Margins as
a Leading
Indicator}\label{monitor-provider-margins-as-a-leading-indicator}

Across scenarios, provider exit in the Onocoy profile is preceded by
sustained deterioration in average provider profitability rather than by
immediate market shocks. This suggests that provider margins function as
a leading indicator of infrastructure risk.

Actionable suggestion:\\
Rather than focusing primarily on provider count or coverage metrics,
Onocoy could incorporate provider margin distributions into internal
monitoring. Tracking how many providers operate near or below breakeven
may provide earlier warning of impending churn than aggregate
participation statistics alone.

This information could inform proactive adjustments to incentives or
cost-sharing mechanisms before infrastructure attrition becomes visible
at the network level.

\subsubsection{Account for Token Price
Sensitivity in Reward
Design}\label{account-for-token-price-sensitivity-in-reward-design}

Liquidity shock scenarios show that price-mediated reward compression
can trigger delayed churn even when nominal rewards remain unchanged.
For Onocoy, where rewards are denominated in ONO, this introduces an
indirect exposure to market volatility.

Actionable suggestion:\\
Onocoy may benefit from evaluating how reward timing, smoothing
mechanisms, or partial stabilization strategies affect provider exposure
to short-term price movements. While full price insulation may be
neither feasible nor desirable, reducing abrupt changes in effective
reward value could help dampen second-order churn effects following
market stress.

This consideration is particularly relevant during known liquidity
events or periods of heightened market uncertainty.

\subsubsection{Differentiate Incentives by
Provider Commitment
Level}\label{differentiate-incentives-by-provider-commitment-level}

Simulation results suggest that providers with higher sunk costs and
longer commitment horizons exhibit greater resilience under stress.
Given the physical and technical requirements of GNSS infrastructure,
Onocoy's provider base is likely heterogeneous in both cost structure
and exit friction.

Actionable suggestion:\\
Onocoy could further explore incentive differentiation based on provider
commitment characteristics, such as hardware tier, installation
complexity, or operational reliability. Reward structures that recognize
long-term contribution may reduce sensitivity to short-term competitive
yield pressure and strengthen infrastructure stability.

This approach aligns incentives with the physical realities of the
network rather than assuming uniform provider behavior.

\subsubsection{Use Stress Testing as an
Ongoing Governance
Tool}\label{use-stress-testing-as-an-ongoing-governance-tool}

One of the central contributions of this thesis is the demonstration of
how stress testing can be applied systematically to DePIN tokenomics.
For Onocoy, such tools need not be limited to academic analysis.

Actionable suggestion:\\
Onocoy could adopt simplified versions of stress-testing frameworks as
part of internal governance or parameter review processes. Periodic
evaluation of how proposed changes perform under adverse scenarios may
help identify unintended consequences before they manifest in
production.

In this sense, stress testing functions less as a predictive instrument
and more as a structured way to reason about risk.

\subsubsection{Positioning
Note}\label{positioning-note}

These suggestions are derived from modeled behavior under standardized
assumptions and should be interpreted as exploratory rather than
definitive. Their value lies in identifying leverage points for further
investigation and experimentation, not in asserting optimal solutions.

\subsection{Implications for Researchers
(What This Enables and What It Does
Not)}\label{implications-for-researchers-what-this-enables-and-what-it-does-not}

This thesis contributes to the growing body of research on Decentralized
Physical Infrastructure Networks (DePIN) by demonstrating how tokenomic
mechanisms can be evaluated through structured stress testing rather
than through growth-oriented or valuation-centric analysis. The
implications for researchers are twofold: the framework introduced here
enables a new mode of comparative analysis, while also highlighting
clear methodological boundaries that future work must respect.

\subsubsection{What This Framework
Enables}\label{what-this-framework-enables}

First, this work provides a reproducible approach for translating
conceptual tokenomic designs into testable models under controlled
conditions. By formalizing incentive mechanisms, provider behavior, and
demand regimes into explicit parameters, researchers can move beyond
narrative comparisons and evaluate directional robustness across
protocols and scenarios.

This enables comparative questions that are otherwise difficult to
answer empirically, such as how different emission structures respond to
identical demand shocks, or how provider retention dynamics vary when
reward--demand coupling weakens. Importantly, the framework allows for
controlled falsification: assumptions can be varied systematically, and
their effects observed transparently.

Second, the stress-testing approach supports interdisciplinary research
at the intersection of economics, systems engineering, and blockchain
governance. DePIN networks operate as socio-technical systems, where
physical constraints, human behavior, and protocol rules interact.
Simulation-based evaluation offers a practical method for studying these
interactions without requiring complete or proprietary real-world data.

Third, the framework is extensible. While this thesis applies the model
to Onocoy as a primary case, the underlying structure can be adapted to
other DePIN categories, alternative incentive designs, or different
demand environments. This positions the work as a methodological
contribution rather than a single-case evaluation.

\subsubsection{What This Framework Does
Not
Enable}\label{what-this-framework-does-not-enable}

At the same time, it is essential to clarify the limits of the approach.
The simulations presented here do not predict real-world outcomes, token
prices, or network success. All results are conditional on modeled
assumptions and should be interpreted as directional indicators rather
than forecasts.

The framework also does not establish causal certainty. While modeled
relationships between incentives, provider behavior, and outcomes are
internally consistent, they remain abstractions of complex real-world
dynamics. External factors such as regulatory changes, technological
breakthroughs, or strategic behavior by large actors are not fully
captured.

Furthermore, the model does not replace empirical validation. Stress
testing can identify plausible failure modes and robustness patterns,
but it cannot confirm whether these dynamics will manifest identically
in live networks. As such, simulation results should be viewed as
complements to empirical observation, not substitutes for it.

\subsubsection{Research Directions Opened
by This
Work}\label{research-directions-opened-by-this-work}

The limitations of this thesis also point toward future research
opportunities. One direction involves integrating richer empirical data,
such as observed provider churn distributions or cost heterogeneity, to
refine behavioral assumptions. Another involves extending the framework
to multi-token or cross-network interactions, which are increasingly
relevant as DePIN ecosystems mature.

Finally, there is scope for methodological refinement. Sensitivity
analysis techniques, alternative agent decision models, and formal
validation against historical network events could further strengthen
the robustness of simulation-based evaluation in this domain.

\subsubsection{Closing Positioning
Note}\label{closing-positioning-note}

Taken together, this thesis positions stress testing as a useful
analytical lens for DePIN tokenomics, while maintaining a clear
separation between evaluation and prediction. By explicitly defining
what the framework enables and where its limits lie, the work aims to
contribute responsibly to an emerging research area characterized by
both technical complexity and high uncertainty.







\subsection{Positioning of This
Contribution}\label{positioning-of-this-contribution}

This thesis positions its contribution at the intersection of mechanism
design, applied simulation, and empirical observation. It does not claim
to predict the success or failure of individual DePIN projects, nor does
it propose a universal template for sustainable tokenomics. Instead, it
offers a disciplined framework for evaluating how tokenomic mechanisms
behave under stress and how those behaviors interact with human
decision-making in real infrastructure networks.

The primary contribution lies in reframing tokenomics evaluation away
from growth narratives and toward resilience analysis. By focusing on
adverse conditions rather than idealized scenarios, the framework
emphasizes directional robustness, failure modes, and trade-offs rather
than optimality. This approach aligns with established practices in
simulation-based research, where the goal is to falsify assumptions and
surface vulnerabilities rather than to optimize parameters in isolation.

A second contribution is methodological. The combination of agent-based
simulation, standardized stress scenarios, and dashboard-aligned metrics
provides a reproducible and extensible evaluation tool. All results are
explicitly conditioned on assumptions, parameter ranges, and scenario
definitions, reinforcing methodological humility and avoiding
overinterpretation. This makes the framework suitable not only for
academic inquiry, but also for practical use by protocol designers
seeking to test incentive structures before irreversible deployment
decisions are made.

Finally, the integration of stress-response archetypes extends DePIN
analysis beyond purely formal mechanisms. By explicitly acknowledging
the role of human intervention under uncertainty, the thesis bridges the
gap between modeled behavior and observed protocol trajectories. This
layered perspective---mechanisms, metrics, and decision
patterns---allows for a more realistic understanding of resilience in
decentralized infrastructure systems.

Taken together, these elements position the thesis as an evaluative
contribution rather than a prescriptive one. It does not assert what
tokenomic designs should be adopted, but it does clarify how different
designs are likely to behave when conditions deteriorate. In doing so,
it aims to improve decision quality before failure occurs, rather than
to explain outcomes after the fact.



\section{Human Decision-Making Under Stress: DePIN Response Archetypes}
\label{sec:human_decision_making}

Table \ref{tab:archetypes} summarizes the five observed stress-response patterns, which are detailed in the following subsections.

\begin{table}[ht]
    \centering
    \small
    \renewcommand{\arraystretch}{1.3}
    \begin{tabularx}{\textwidth}{@{} 
        >{\raggedright\arraybackslash\hsize=0.8\hsize}X 
        >{\raggedright\arraybackslash\hsize=0.6\hsize}X 
        >{\raggedright\arraybackslash\hsize=1.4\hsize}X 
        >{\raggedright\arraybackslash\hsize=1.2\hsize}X 
    @{}}
        \toprule
        \textbf{Archetype} & \textbf{Primary Driver} & \textbf{Tokenomic Consequence} & \textbf{Key Signal} \\
        \midrule
        \textbf{I: Subsidy Inertia} & Retention Fear & Widens subsidy gap; dilutes provider revenue. & Declining burn-to-emission. \\
        \addlinespace
        \textbf{II-A: Subsidy Boosting} & Short-term Churn Fix & Accelerates insolvency; creates ``incentive overfitting''. & Temporary retention spike $\to$ crash. \\
        \addlinespace
        \textbf{II-B: Incentive Re-Targeting} & Efficiency Alignment & Preserves core solvency; squeezes marginal supply. & Improved utilization efficiency. \\
        \addlinespace
        \textbf{III: Narrative Pivot} & Political Evasion & Delays adjustment; decouples price from utility. & Stable messaging vs bad metrics. \\
        \addlinespace
        \textbf{IV: Emergency Centralization} & Survival & Suspends decentralization; externalizes governance risk. & Rising concentration. \\
        \bottomrule
    \end{tabularx}
    \caption{Human Decision-Making Archetypes Under Stress}
    \label{tab:archetypes}
\end{table}



\subsection{Why Mechanism Design Alone Is Insufficient}
Tokenomic mechanisms do not operate in isolation. In real-world DePIN deployments, protocol behavior emerges from the interaction between formal incentive rules and human decision-making under uncertainty. While simulation-based evaluation can illuminate how mechanisms respond to stress in a controlled environment, actual network trajectories are shaped by how teams interpret signals, prioritize risks, and intervene---often under time pressure and incomplete information.

Existing DePIN analyses tend to abstract away this human layer, implicitly assuming that protocol parameters evolve optimally or that governance reacts smoothly to deteriorating conditions. In practice, responses to stress are neither instantaneous nor purely rational. Instead, they follow recurring behavioral patterns that interact with tokenomic structures in predictable ways.

This section introduces a set of stress-response archetypes observed across DePIN projects during periods of adverse conditions \cite{Messari2024}. These archetypes are not judgments of competence or intent. Rather, they represent recurrent decision patterns that emerge when teams attempt to stabilize networks facing declining demand, price shocks, or provider churn. Importantly, each archetype interacts differently with tokenomic mechanisms and can either mitigate or amplify structural fragility.

\subsection{Archetype I: Subsidy Inertia (Emissions Denial)}
\textbf{Definition:} Subsidy inertia refers to the continued reliance on emissions-driven provider rewards despite clear signals of demand contraction or declining incentive solvency.
\newline
\textbf{Rationale at the Time:} This behavior is often motivated by short-term retention concerns. For infrastructure-heavy DePINs like Onocoy, provider exit is costly and partially irreversible. Teams fear that lowering rewards will signal weakness or trigger irreversible provider exits. Emissions are thus treated as a stabilizing force rather than as a source of dilution.
\newline
\textbf{Interaction with Tokenomics:} Under sustained demand weakness, emissions denial increases the gap between token issuance and usage-driven sinks. Simulation results show that this widens inflationary pressure and compresses real provider profitability, even when nominal rewards remain constant. The system appears stable in the short term but accumulates hidden fragility.
\newline
\textbf{Dashboard Signals:}
\begin{itemize}
    \item Declining burn-to-emission ratio.
    \item Increasing token velocity driven by cost-covering sales.
    \item Rising provider losses despite unchanged nominal rewards.
\end{itemize}

\subsection{Archetype II: Incentive Overfitting vs. Re-Targeting}
This archetype splits into two materially different responses.

\subsubsection{II-A: Subsidy Boosting (Overfitting)}
\textbf{Definition:} Broad increases in rewards or temporary bonus programs designed to arrest churn without altering what is being rewarded.
\newline
\textbf{Interaction with Tokenomics:} While incentive extensions can temporarily reduce churn, simulations demonstrate that they often worsen long-term sustainability by increasing emissions without addressing demand-side constraints. Over time, providers become conditioned to elevated subsidies, making later normalization more disruptive.
\newline
\textbf{Dashboard Signals:}
\begin{itemize}
    \item Short-lived improvements in retention metrics.
    \item Worsening solvency ratios.
    \item Increased sensitivity to subsequent shocks.
\end{itemize}

\subsubsection{II-B: Incentive Re-Targeting (Adaptive Alignment)}
\textbf{Definition:} Modification of reward logic to better reflect economically valuable output rather than participation volume.
\newline
\textbf{Onocoy Context:} Onocoy's focus on quality, availability, and location scales reflects an early commitment to incentive re-targeting. The quadratic decay applied to dense geographic clustering serves as a structural guardrail against over-rewarding redundant supply.
\newline
\textbf{Dashboard Signals:}
\begin{itemize}
    \item Improved utilization efficiency.
    \item Slower provider growth but higher average profitability.
    \item More stable burn-to-emission trajectories under demand volatility.
\end{itemize}

\subsection{Archetype III: Narrative Pivot Without Structural Adjustment}
\textbf{Description:} Rather than modifying tokenomic parameters, teams shift messaging toward future use cases, long-term vision, or upcoming integrations. The underlying incentive structure remains largely unchanged.
\newline
\textbf{Rationale at the Time:} Narrative pivots are low-cost and reversible. They are often employed when governance processes are slow or when parameter changes are perceived as risky or controversial.
\newline
\textbf{Interaction with Tokenomics:} Narrative changes have limited impact on provider economics when structural conditions remain unchanged. Simulation results indicate that, absent demand recovery, incentive solvency continues to deteriorate regardless of narrative framing.

\subsection{Archetype IV: Emergency Centralization}
\textbf{Description:} Under severe stress, extensive intervention leads to temporary or permanent centralization of control, infrastructure, or execution environments to preserve network functionality.
\newline
\textbf{Comparative Evidence:} The migration of Helium to Solana represents a clear instance where operational survivability was prioritized over architectural purity.
\newline
\textbf{Interaction with Tokenomics:} Emergency interventions can stabilize capacity and reduce churn in the short term. However, they introduce governance risk and undermine long-term decentralization incentives. Simulations suggest that while collapse probabilities decrease, recovery elasticity often diminishes.

\subsection{Implications for Onocoy (Diagnostic Alignment)}
The primary contribution of these archetypes is not explanatory storytelling, but diagnostic alignment. Each response pattern maps directly to observable signals within the stress-testing dashboard developed in this thesis.

Rather than prescribing specific decisions, the framework allows Onocoy's operators to ask a fundamental question: \textit{Which stress-response pattern are we currently exhibiting?}
For Onocoy, this framing clarifies how specific design choices---such as emission decay and location penalties---interact with human decision-making to shape resilience trajectories.

\subsection{Counterexamples, Edge Cases, and Deliberate Exclusions}
The analytical framework presented in this thesis emphasizes comparative robustness of DePIN tokenomic mechanisms under stress. As with any applied modeling approach, its explanatory power depends on clearly defined boundaries.

\subsubsection{Counterexample I: Demand-Dominant DePIN Regimes}
A first class of counterexamples arises in DePINs that successfully reach a demand-dominant regime, where exogenous usage growth overwhelms tokenomic stress dynamics. In such cases, increasing service demand drives burn mechanisms and provider revenue independently of short-term incentive tuning. Under these conditions, archetypes like subsidy inertia become less predictive. This distinction reinforces the thesis scope: the analysis is designed for early-to-mid-stage DePINs operating under demand uncertainty.

\subsubsection{Counterexample II: Structurally Centralized Networks}
Networks relying on persistent organizational centralization for incentive coordination are excluded by design. Their stability is organizational rather than tokenomic, governed by administrative authority rather than decentralized feedback loops.

\subsubsection{Excluded Archetypes and Metrics}
\textbf{Governance Paralysis:} While common, this represents a failure to respond rather than a stable archetype.
\newline
\textbf{Speculative Reflexivity:} Dropped to avoid conflating price momentum with infrastructure utility.
\newline
\textbf{Excluded Metrics:} Raw token price and market cap were excluded due to high volatility and weak interpretability under stress \cite{CFA2018}. The analysis prioritizes metrics like provider retention and utilization efficiency.

\newpage

% Appendix
\appendix
\appendix
\section{Appendix}
\label{sec:appendix}

% Appendix content goes here

\newpage

% Bibliography
\printbibliography

\end{document}
